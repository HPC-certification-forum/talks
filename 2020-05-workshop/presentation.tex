\PassOptionsToPackage{hyphens}{url}
\documentclass[compress,aspectratio=169]{beamer}

\usetheme{Reading}

\graphicspath{{../2019-06-isc/}{../2019-06-isc/fig/}{img/}{../logo/}}

\newcommand{\ok}[1]{{#1 (done)}}
\newcommand{\ongoing}[1]{{#1 (ongoing)}}
\newcommand{\started}[1]{{#1 (started)}}
\newcommand{\pending}[1]{{#1 (pending in plan)}}
\newcommand{\hrefb}[2]{\href{#1}{\textcolor{blue}{#2}}}

\subtitle{}
\title{\Large Toward a Globally Acknowledged and Free HPC Certification}
\author{Julian Kunkel (+ HPC Certification Forum)}
\date{2020-05-20}
\authorURL{https://hpc-certification.org}
\authorFooter{Julian M. Kunkel et al.}
\venue{HPCCF Virtual Workshop}
\institute{Department of Computer Science}
\groupLogo{\includegraphics[width=2.5cm]{hpccf-small}}
\titleLogo{ \includegraphics[height=2.5cm]{blur-book-stack-books-590493}\includegraphics[height=2.5cm]{ground-group-growth-461049}\includegraphics[height=2.5cm]{accomplishment-ceremony-college-267885}\includegraphics[height=2.5cm]{ground-group-growth-461049}\includegraphics[height=2.5cm]{blur-book-stack-books-2}}


\begin{document}

\begin{frame}[plain]{}
	\maketitle
\end{frame}


%       the nature of the training or education program
%       Strategy
%       assessment or evaluation technique
%       situations for which it is relevant or in which it was applied
%       an evaluation of its success
%       lessons learned
%       reproducibility of the processes and resources

\section{The Workshop}

\begin{frame}{Workshop}
  \begin{block}{Goals}
    \begin{itemize}
      \item Establish globally acknowledged HPC certification
        \begin{itemize}
          \item Discuss opportunities and roadmap, foster collaboration
        \end{itemize}
    \end{itemize}
  \end{block}

  \vspace*{-0.7em}


  \begin{block}{Agenda}
    \begin{itemize}
      \item Introduction to the HPC Certification Forum (20 min)
      \item Invited speakers (10 min each)
      \item Examination and certification (20 min)
      \item Discussion
    \end{itemize}
  \end{block}

  \vspace*{-0.3em}

  \begin{block}{Interactivity}
    \begin{itemize}
      \item Q\&A time slot after each talk
      \item Please feel free to ask questions ASAP in the chat
      \item Critical discussions are welcome!
    \end{itemize}
  \end{block}

\end{frame}


\section{The Forum}
\sectionIntroHidden

\subsection{}


\begin{frame}{Challenges for HPC Training}
		\begin{itemize}
			\item Not all users possess the right level of training
				\begin{itemize}
				\item Inefficient usage of systems, frustration, lost potential
				\item Good training saves compute time and costs!
				\end{itemize}
      \item Diverse user background and goals
        \begin{itemize}
          \item Science is the goal, HPC is the vehicle
          \item Need to run an application to complete the PhD
        \end{itemize}
      \item Learning is not easy
			\begin{itemize}
				\item Users need to understand beneficial knowledge for tasks
				\item There exist various different training material
				\item Teaching of different data centers is hard to compare
			\end{itemize}
			\item Data center have difficulties to verify the skills of users
		\end{itemize}

\end{frame}


\begin{frame}{The HPC Certification Program}
		\begin{block}{Goals}
			\begin{itemize}
				\item Fine-grained standardizing HPC knowledge representation
          \begin{itemize}
            \item What competences exist, how are they defined?
            \item Puzzle of competences for everyone (practitioners, students)
            \item Supporting navigation and role-specific knowledge maps
          \end{itemize}
				\item Establishing international certificates attesting knowledge
			\end{itemize}
		\end{block}

		\begin{block}{Important!}
			\begin{itemize}
				\item We do not compete with content providers
				\item We do not intent to create a curriculum
			\end{itemize}
		\end{block}

    \begin{block}{This talk is about contributing to the knowledge representation}
    \end{block}
\end{frame}



\begin{frame}{Classification of HPC Competences}
	\begin{itemize}
		\item Organization of HPC skills
		\begin{itemize}
			\item Skills are typically depending on sub-skills $\Rightarrow$ tree structure
			\item References to skills are possible; still skills are building blocks for various tasks
			\item One skill can have multiple instances for different skill levels
		\end{itemize}

		\item Granularity of skill descriptions
		\begin{itemize}
			\item Too fine $\Rightarrow$ content of a skill is predefined at leaf level
			\item Too coarse $\Rightarrow$ no help for structuring the material
			\item Guiding principle: leaf node should be coverable in 2-4 hour lecture/workshop
		\end{itemize}

		\item External information can be linked to the skills providing different \textbf{views}
		\begin{itemize}
			\item Suitability for a user role (Tester, Builder, Developer)
			\item Suitability for a scientific domain (Chemistry, Physics, ...)
			\item View: purpose-specific representation / coloring / content
				\begin{itemize}
				\item Groups/institutions can derive a new skill tree with their own emphasis
				\item  What should people know to effectively work in your environment?
				\end{itemize}
		\end{itemize}
	\end{itemize}
\end{frame}



\begin{frame}{Further Considerations}
	\begin{itemize}
		\item Certificate definition
		\begin{itemize}
			\item Bundles a set of useful skills together %(e.g. "Getting startet with HPC Clusters")
			\item A users' HPC qualification is certified by successful exams
		\end{itemize}
		\item Separation of skill, certificates and content provider
		\begin{itemize}
			\item Similar to the concept of a high school graduation exam %("Zentralabitur")
			\item Learning material can be provided by different institutions
			\item Teachers can put badges on material: this "trains XYZ"
		\end{itemize}
		\item Verification of skill tree and certification approach
			\begin{itemize}
				\item We utilize the HPC community/practitioners to justify approaches
			\end{itemize}
	\end{itemize}
\end{frame}


\begin{frame}{Status}

\begin{columns}
\column{0.8\textwidth}
	\begin{itemize}
	\item Organizing regular meetings (see our webpage)
	\item Released a first skill tree (we will discuss this after the talks)
	\item Released technical representations of the HPC skills
	\item Released JavaScript for visualization of skill tree \hrefb{https://www.hpc-certification.org/skills/}{(demo)}
		\begin{itemize}
			\item Enables views: adjustable/embedable in your webpage
		\end{itemize}
	\item Developed prototype for exam process: legal framework
	\item Designed seal of endorsement
	\item Engaged with various stakeholders (e.g., SIGHPC Edu)
	\item Conducted survey to verify the skill tree (more to come!)
\end{itemize}

\column{0.2\textwidth}
\includegraphics[width=\textwidth]{certified.pdf}
\end{columns}

\medskip

\textit{All our developments are under open licenses (except the exam questions)}
\end{frame}


\begin{frame}{The \includegraphics[width=0.45\textwidth]{hpccf-full}}
	The HPC-CF is the central authority for the development of the program

	\begin{block}{Organization Details}
		\begin{itemize}
			\item An independent international body
			\item Organized into
				\begin{itemize}
					\item Steering board
					\item Full members with voting rights
					\item Associate members
          \item Collaboration with e.g., SIGHPC Education Chapter
				\end{itemize}
		\end{itemize}
	\end{block}

	\begin{block}{Responsibilities}
		\begin{itemize}
			\item Curating and maintaining the skill tree and certificates
			\item Providing tools and ecosystem around the competences
		\end{itemize}
	\end{block}
\end{frame}

\section{Skills}
\sectionIntroHidden

\begin{frame}{Content of the Certification Program}
	\begin{itemize}
		\item A \textbf{skill} defines background, objectives, learning outcomes
		\item The \textbf{skill tree} organizes the competences as hierarchical skills
		\item Certificates bundle several skills into attestable unit
	\end{itemize}

	\begin{figure}
		\includegraphics[width=\textwidth]{skill-tree}
		\vspace*{-2em}
		\caption{Top-levels of the skill tree (We are working on ADM and BDA branches)}
	\end{figure}
\end{frame}


\begin{frame}{Example High-Level Skill}
\begin{itemize}
\item Name: SLURM Workload manager
\item Id: USE4.2.2-B
\item Background: {\small SLURM is a widely used open-source workload
manager providing various advanced features.}
\item Aim:
\begin{itemize}
\item comprehend and describe the basic architecture of SLURM and its tools
\item use relevant tools to run and monitor (parallel) applications
\end{itemize}
\end{itemize}

\begin{block}{Learning outcomes}
\begin{itemize}
\item run interactive jobs with salloc, a batch job with sbatch
\item explain the architecture of SLURM, i.e., the role of slurmd, srun
\item explain the function of the tools: sacct, sbatch, salloc, ...
\item explain time limits and the benefit of a backfill scheduler
\item see \url{https://www.hpc-certification.org/wiki/}
\end{itemize}
\end{block}
\end{frame}




\section{Contributing}

\begin{frame}{High-Level Editing}
  \begin{itemize}
    \item Webpage with Markdown version controlled in Git
      \begin{itemize}
        \item \url{https://www.hpc-certification.org/wiki/skill-tree/b}
        \item GitHub: \url{https://github.com/HPC-certification-forum/skill-tree}
      \end{itemize}
    \item Editing a MindMap, the structure of Skills
      \begin{itemize}
        \item Synchronized with the skill tree in Git
        \item Uses the OpenSource tool Freemind
      \end{itemize}
    \item Discussion on our \href{https://join.slack.com/t/hpc-certification/shared_invite/enQtMzUwNzU3NzM2MTkzLTAzZWM3NDg0N2I2ZmQwOWI5ZGUwNjNlNDgzM2RmOTM3ZWRjNjIxYTc5NzUxYTJhNmRlNmM5YmE1NDY3YzkzYzA}{Slack}
    \item We welcome any contribution via either channel
      \begin{itemize}
        \item Pull requests are also welcome
      \end{itemize}
  \end{itemize}
\end{frame}



\section{Conclusions}
\sectionIntroHidden

\begin{frame}{Outlook and Expected Benefits}
	\begin{block}{HPC practitioners}
		\vspace*{-0.2cm}
	\begin{itemize}
	\item Increase motivation to participate as the certificates are recognized in a CV
	\item Validate knowledge via tests
	\item Browse relevant competences
	\item Identify recommended and required skills related to certain tasks
	\item Understand and compare teaching offers across sites
	\end{itemize}
	\end{block}
	\vspace*{-0.3cm}
	\begin{block}{Data centers}
		\vspace*{-0.2cm}
	\begin{itemize}
	\item Increase sharing of teaching materials
	\item Simplifies documentation of taught skills
	\item Identify missing teaching activities
	\item Tailor skill-representation specifically to users
	\item Correlate lack of skills with efficient use
	\end{itemize}
	\end{block}
\end{frame}



\begin{frame}{Summary}

	\begin{block}{HPC Certification Program}
		\begin{itemize}
			\item Effort to standardize representation/certification of relevant HPC skills
      \begin{itemize}
        \item Hierarchical definition of skills for practitioners
        \item Building blocks that can be cherry-picked for different tasks
				\item It's goal is \textbf{NOT} to provide content or a linear curriculum
      \end{itemize}
			\item Perspective for data centers
				\begin{itemize}
					\item Use statistics and machine learning to direct users to right skills
					\item Make certain skills a mandatory requirement?
				\end{itemize}
			\item Customizable representation and navigation for data centers/domains
      \begin{itemize}
        \item Interactive viewer to browse skills and related content
				\item We will use the viewer to link good content to the skills, too!
      \end{itemize}
      \item Visit us and join our Slack/mailing lists: \url{https://hpc-certification.org}
		\end{itemize}
	\end{block}
\end{frame}




\end{document}

%-------------------------------------------------------%
%      HPCCF Virt. Workshop Presentation CM             %
%-------------------------------------------------------%
\documentclass[english,xcolor=pdftex,dvipsnames,compress,aspectratio=169]{beamer}


%\setbeamertemplate{mini frames}[box]
\usepackage{babel}
\usepackage[utf8]{inputenc}
\usepackage[T1]{fontenc}
\usepackage{amsfonts,amsmath,amssymb}
\usepackage{wrapfig}
\usepackage{pifont}

\usepackage{color,colortbl}
\usepackage{upquote}
% \usepackage{showexpl}
% \lstset{
%     basicstyle=\ttfamily\small,
%     commentstyle=\itshape\ttfamily\small,
%     showspaces=false,
%     showstringspaces=false,
%     breaklines=true,
%     breakautoindent=false,
%     captionpos=t
% }

\definecolor{pblue}{RGB}{45,106,148}
\definecolor{pdarkblue}{RGB}{35,71,100}
\definecolor{plightblue}{RGB}{90,159,212}
\definecolor{pyellow}{RGB}{255,212,59}
\definecolor{pdarkyellow}{RGB}{255,188,41}
\definecolor{orange}{RGB}{255,165,0}
\definecolor{plightyellow}{RGB}{255,232,115}
\definecolor{pdarkgrey}{RGB}{100,100,100}
\definecolor{pgrey}{RGB}{153,153,153}
\definecolor{plightgrey}{RGB}{233,233,233}
\definecolor{plightgrey2}{RGB}{247,247,247}
\definecolor{pnavy}{RGB}{0,0,170}
\definecolor{BrickRed}{RGB}{150,22,11}
\definecolor{BlueViolet}{RGB}{138, 43, 226}
\definecolor{PineGreen}{RGB}{0, 51, 0}
\definecolor{light-gray}{gray}{0.95}

\definecolor{UniRot}{RGB}{193,0,42}
\definecolor{UniDunkelGrau}{RGB}{99,99,99}
\definecolor{UniHellGrau}{RGB}{172,172,172}

\definecolor{UrlColor}{rgb}{0,0.08,0.45}
\definecolor{links}{rgb}{0,0,0}

\usetheme{CambridgeUS} % Pittsburgh, CambridgeUS
\usecolortheme{beaver} %wolverine | crane | beaver | seahorse
\useinnertheme{rounded} 
\useoutertheme{default}
\usefonttheme{default}
%\setbeamercovered{transparent}
\setbeamertemplate{footline}[frame number]
\setbeamersize{text margin left=0.5cm, text margin right=0.5cm}

\setbeamercolor{structure}{fg=UniRot}% to modify  immediately all palettes
\setbeamercolor{title}{fg=UniRot}
\setbeamercolor{title in head/foot}{fg=UniRot}

\setbeamercolor{block title}{bg=UniRot!20,fg=darkred}
\setbeamercolor{block body}{fg=black, bg=plightgrey2}

% \setbeamercolor{block title}{fg=white,bg=orange}
\setbeamercolor{block title alerted}{fg=white,bg=UniRot}
\setbeamercolor{block title example}{fg=white,bg=PineGreen!80}

\graphicspath{{../2019-06-isc/}{../2019-06-isc/fig/}{img/}{../logo/}{images/}}

\usepackage{tikz}
\usetikzlibrary{arrows,shapes,backgrounds,positioning,shadows,decorations,trees,decorations.pathreplacing}


\addtobeamertemplate{footline}{}{%
\begin{tikzpicture}[remember picture,overlay]
\node[anchor=south west,yshift=2pt] at (current page.south west) {\includegraphics[height=0.8cm]{./images/zdv_logo.png}};
\end{tikzpicture}}

\usepackage[tikz]{bclogo}
\newcommand{\task}[2][Over to you]{\begin{bclogo}[arrondi=0.1,logo=\bcoutil]{#1} #2 \end{bclogo}}
\newcommand{\exercise}[2][ ]{\begin{bclogo}[arrondi=0.1,logo=\bcoutil]{Excercise -- Type: #1}  #2 \end{bclogo}}
\newcommand{\docs}[2][Documentation]{\begin{bclogo}[arrondi=0.1,logo=\bcplume]{#1} #2 \end{bclogo}}
\newcommand{\hint}[2][Hint]{\begin{bclogo}[arrondi=0.1,logo=\bcinfo]{#1} #2 \end{bclogo}}
\newcommand{\warning}[2][Warning]{\begin{bclogo}[arrondi=0.1,logo=\bcattention]{#1} #2 \end{bclogo}}
% ``d/Definition'' is already defined ;-)
\newcommand{\explanation}[2][Definition]{\begin{bclogo}[arrondi=0.1,logo=\bcplume]{#1} #2 \end{bclogo}}
\newcommand{\question}[2][Question]{\begin{bclogo}[arrondi=0.1,logo=\bcquestion]{#1} #2 \end{bclogo}}


\subtitle{HPCCF Virtual Workshop}
\title{\Large Certification Strategy and Contributions}
\author{Christian Meesters (+ HPC Certification Forum)}
\date{2020-05-18}
%\authorURL{https://hpc-certification.org}
%\authorFooter{Christian Meesters \& Julian Kunkel}
%\venue{HPCCF Virtual Workshop}
\institute{HPC Group -- Johannes Gutenberg-University of Mainz}
%\groupLogo{\includegraphics[width=2.5cm]{hpccf-small}}

\usepackage{multicol}

\usepackage{hhline}

\usepackage{times}

% will decrease the font size for one frame
\newcommand\Fontvi{\fontsize{6}{7.2}\selectfont}

\usepackage{verbatim}
\usepackage{listings}

\lstloadlanguages{Python,bash,C++}
\lstset{showspaces=false,
basicstyle=\small,
showstringspaces=false}


%default python listings:
\lstdefinestyle{Python}
{
  language=Python,
  basicstyle=\small,
  showstringspaces=false,
  stepnumber=5,
  numberstyle=\tiny,
  numbersep=5pt,
  showspaces=false,
  frame=single,
  framerule=0.4pt,
  rulecolor=\color{pgrey},
  backgroundcolor=\color{white},
  stringstyle=\color{BrickRed},
  keywordstyle=\color{BlueViolet}\bfseries,
  commentstyle=\color{PineGreen}\bfseries,
  identifierstyle={},
  emph={[10]self}, emphstyle={[10]\color{pblue}},
  emph={[11]yield}, emphstyle={[11]\color{pblue}},
}

%default python listings:
\lstdefinestyle{C++}
{
  language=C++,
  basicstyle=\small,
  showstringspaces=false,
  stepnumber=5,
  numberstyle=\tiny,
  numbersep=5pt,
  showspaces=false,
  frame=single,
  framerule=0.4pt,
  rulecolor=\color{pgrey},
  backgroundcolor=\color{white},
  stringstyle=\color{BrickRed},
  keywordstyle=\color{BlueViolet}\bfseries,
  commentstyle=\color{PineGreen}\bfseries,
  identifierstyle={},
  emph={[10]self}, emphstyle={[10]\color{pblue}},
  emph={[11]yield}, emphstyle={[11]\color{pblue}},
}

\newcommand{\CC}{C\nolinebreak\hspace{-.05em}\raisebox{1ex}{\tiny\bf +}\nolinebreak\hspace{-.10em}\raisebox{1ex}{\tiny\bf +}}

%default shell listings:
\lstdefinestyle{Shell}
{
  language=Bash,
  basicstyle=\ttfamily\small,
  showstringspaces=false,
  frame=single,
  framerule=0.4pt,
  rulecolor=\color{pgrey},
  backgroundcolor=\color{plightgrey2},
  stringstyle=\color{BrickRed},
  keywordstyle=\color{BlueViolet},
  commentstyle=\color{PineGreen}\bfseries,
  identifierstyle=\color{black},
  emph={[10]\$,>>>}, emphstyle={[10]\color{pblue}},
  moredelim=**[is][\bfseries\color{red}]{@}{@}
}

%default plain listings (e.g. for config files):
\lstdefinestyle{Plain}
{ 
  stepnumber=5,
  numberstyle=\tiny,
  numbersep=5pt,
  language=Bash,
  basicstyle=\ttfamily\small,
  showstringspaces=false,
  frame=single,
  framerule=0.4pt,
  rulecolor=\color{pgrey},
  backgroundcolor=\color{plightgrey2},
  stringstyle=\color{black},
  keywordstyle=\color{black},
  commentstyle=\color{blue}\bfseries,
  identifierstyle=\color{black},
  emph={[10]\$,>>>}, emphstyle={[10]\color{pblue}}
}
\lstdefinelanguage{XML}
{
  frame=single,
  framerule=0.4pt,
  rulecolor=\color{pgrey},
  backgroundcolor=\color{plightgrey2},
  stringstyle=\color{black},
  keywordstyle=\color{black},
  commentstyle=\color{blue}\bfseries,
  identifierstyle=\color{black},
  emph={[10]\$,>>>}, emphstyle={[10]\color{pblue}}
  morestring=[b]",
  morestring=[s]{>}{<},
  morecomment=[s]{<?}{?>},
  morekeywords={xmlns,version,type}% list your attributes here
}

\newcommand{\bibtex}{\textsc{Bib}\TeX}

%%% https://tex.stackexchange.com/questions/99316/symbol-for-external-links
\newcommand{\LinkSymbol}{%
  \tikz[x=1.2ex, y=1.2ex, baseline=-0.05ex]{% 
    \begin{scope}[x=1ex, y=1ex]
      \clip (-0.1,-0.1) 
      --++ (-0, 1.2) 
      --++ (0.6, 0) 
      --++ (0, -0.6) 
      --++ (0.6, 0) 
      --++ (0, -1);
      \path[draw, 
      line width = 0.5, 
      rounded corners=0.5] 
      (0,0) rectangle (1,1);
    \end{scope}
    \path[draw, line width = 0.5] (0.5, 0.5) 
    -- (1, 1);
    \path[draw, line width = 0.5] (0.6, 1) 
    -- (1, 1) -- (1, 0.6);
  }
}
\newcommand{\lhref}[2]{\href{#1}{#2\,\LinkSymbol}}

%%%% shortcuts for uniform appearance of common strings %%%%
\newcommand{\slurm}{\textsc{slurm}~}
\makeatletter
\newcommand{\rmnum}[1]{\romannumeral #1}
\newcommand{\Rmnum}[1]{\expandafter\@slowromancap\romannumeral #1@}
\makeatother
\usepackage{xspace}
\newcommand{\mogon}{\textsc{mogon}\xspace}
\newcommand{\mogonI}{\textsc{mogon}\,\Rmnum{1}\xspace}
\newcommand{\mogonII}{\textsc{mogon}\,\Rmnum{2}\xspace}

\setcounter{tocdepth}{1}


%%%%%%%%%%%%%%%%%%%%%%%%%%%%%%%%%%%%%%%%%%%%%%%%%%%%%%%%%%%%%%%%%%%%%%%%%%%%%%%%
%%%%%%%%%%%%%%%%%%%%%%%%%%%%%%%%%%%%%%%%%%%%%%%%%%%%%%%%%%%%%%%%%%%%%%%%%%%%%%%%
\begin{document}
%%%%%%%%%%%%%%%%%%%%%%%%%%%%%%%%%%%%%%%%%%%%%%%%%%%%%%%%%%%%%%%%%%%%%%%%%%%%%%%%
%%%%%%%%%%%%%%%%%%%%%%%%%%%%%%%%%%%%%%%%%%%%%%%%%%%%%%%%%%%%%%%%%%%%%%%%%%%%%%%%

% Passe captions an
\setbeamertemplate{caption}{\insertcaption}
% \setbeamerfont{caption}{size=\scriptsize}
\setlength\abovecaptionskip{-2.5pt}
\setlength\belowcaptionskip{0pt}



% For every picture that defines or uses external nodes, you'll have to
% apply the 'remember picture' style. To avoid some typing, we'll apply
% the style to all pictures.
\tikzstyle{every picture}+=[remember picture]
\tikzstyle{na} = [baseline=-.5ex]

%%%%%%%%%%%%%%%%%%%%%%%%%%%%%%%%%%%%%%%%%%%%%%%%%%%%%%%%%%%%%%%%%%%%%%%%%%%%%%%% 
\begin{frame}[plain] % plain erzeugt Titelseite ohne Kopf- und Fußzeile
  \titlepage
\end{frame}

\section[Certification Strategy]{strategy}

\subsection[What's in for Users]{Users}

\begin{frame}
  \frametitle{How the Site Manager looks on HPC Education}
  \begin{columns}
   \begin{column}{0.6\textwidth}
    \centering
    \includegraphics[width=0.8\textwidth]{images/runner}
   \end{column}
   \begin{column}{0.4\textwidth}
    \pause
    \begin{itemize}[<+->]
     \item ressources are always limited
     \item teaching ressources even more
     \item integration into HPCCF might offer more (still needed) courses
    \end{itemize}
   \end{column}
  \end{columns}
\end{frame}


\begin{frame}
  \frametitle{How Joe User looks on HPC}
  \begin{columns}
   
  \begin{column}{.5\textwidth}
    \centering
  \includegraphics[width=0.8\textwidth]{images/joe}
   \end{column}
  

   \begin{column}{.5\textwidth}
    \pause
    Most users
    \begin{itemize}[<+->]
     \item \ldots use $3^{rd}$ party applications \ldots
     \item \ldots will need (yet not always visit) an intro course \ldots
     \item \ldots perhaps a scripting course \ldots
     \item \ldots only \emph{really} interested in workflows taylored for their need.
    \end{itemize}
    \pause
    And will never leave their site for other HPC courses!
   \end{column}
   \end{columns}

%   \centering
%   \includegraphics[width=0.4\textwidth]{images/joe}
\end{frame}

\begin{frame}
  \frametitle{How Bruce Coder looks on HPC}
  \begin{columns}
   
  \begin{column}{.6\textwidth}
    \centering
  \includegraphics[width=0.8\textwidth]{images/poweruser}
   \end{column}
  

   \begin{column}{.4\textwidth}
    \pause
    Only power users 
    \begin{itemize}[<+->]
     \item \ldots will select \emph{their} topics \ldots
     \item \ldots will care to travel for computing topics \ldots
     \item \ldots will rarely need intro courses \ldots
    
    \end{itemize}

   \end{column}
   \end{columns}
\end{frame}

\subsection{Certification Process}

\begin{frame}
  \frametitle{Strategy}
  
\end{frame}



\section[Designing Questions]{design}

\begin{frame}
  \frametitle{Disclaimer}
  Some examples are inspired by Greg Wilsons book
 \begin{center}
  \sf{Teaching Tech Together} (CRC Press, 2020)
 \end{center}
 Some ideas are based on own experience, some on other sources.
\end{frame}

\subsection{Question Types}

\begin{frame}
  \frametitle{Purposes \ldots}
  Before diving into Question Design, note:
  \begin{itemize}
    \item a question can be asked with a certain aim
    \item different courses ask for different knowledge / skills
    \item $\curvearrowright$ questions need to be designed and choosen with care
  \end{itemize}
\end{frame}


\begin{frame}
  \frametitle{What is in the arsenal?}
  \centering{\scriptsize
  \begin{tabular}{llll}
     
     Type & Good for testing \ldots & Comment & Status\\\hline
     Multiple Choice & Conceptions & already given& {\color{PineGreen}\checkmark}\\\pause
     Fill in the Blanks & Conceptions \& Knowledge & for novices & {\color{BrickRed}\bf\ding{55}}\\
     Parsons Problem & coding skills & tools exists online & {\color{BrickRed}\bf\ding{55}}\\
     Tracing Execution Order & debugging skills  & (can be MCQ) & {\color{PineGreen}\checkmark}\\
     Reverse Execution & deductive reasoning & can be fill in the blanks & {\color{BrickRed}\bf\ding{55}}\\
     Labeling Diagrams & judgement skills & can be implemented as MCQ & {\color{PineGreen}\checkmark}\\
 \end{tabular}}
 \hint[Note]{This list is not intended to be a comprehensive list!}
\end{frame}


% overview
% background selection

\subsection{Multiple Choise Questions}

\begin{frame}
 \frametitle{Multiple Choice -- When?}

 Multiple Choice Questions (MCQs) are popular when designing e-learning tests \ldots\vspace{-1em}
 \pause
 \newline
 \only<2->{\question{When are they most suitable?}}
 \pause
 Suppose you are teaching children and you give them this MCQ:
 \exercise[Testing Conceptions]{
  What is 37 + 15?
  \begin{enumerate}[a)]
   \item 52  {\color{pdarkgrey}correct}
   \item 42  {\color{pdarkgrey}child did not understand ``carrying''}
   \item 412 {\color{pdarkgrey}child treated every column seperately}
   \item 43  {\color{pdarkgrey}knows she has to carry 1, but to wrong column}
  \end{enumerate}
 }
\end{frame}


\begin{frame}
 \frametitle{Multiple Choice -- When? (continued)}
 
 The Young-Child question rephrased for newbies to the SLURM batch system:
 \vspace{-1em}
 \exercise[Testing Conceptions about SLURM]{
   Think of a cluster with 20 core nodes. If a job is submitted with the following parameterisation, how many nodes are reserved?\newline
   \texttt{\#SBATCH -n 20}\newline
   \texttt{\#SBATCH -c 2}
   \begin{enumerate}[a)]
    \item 2 {\color{pdarkgrey}correct}
    \item 4 {\color{pdarkgrey}user did correctly multiply, but is not aware of the 20 cores}
    \item 1 {\color{pdarkgrey}user did not multiply by \texttt{-c 2}}
    \item unkown without \texttt{N}-flag {\color{pdarkgrey}user did not understand the concept}
   \end{enumerate}
  }
\end{frame}



\section[Contributing to the Question Pool]{Contributing}

\begin{frame}
  \frametitle{Contributions via the HPCCF-Wiki}
  \centering
  \includegraphics[width=0.8\textwidth]{images/contribution}
\end{frame}


\begin{frame}
  \frametitle{Contributions via the HPCCF-Wiki II}
  Each \lhref{https://www.hpc-certification.org/wiki}{HPCCF wiki} page contains a link. It leads to a little form asking for:
  \begin{itemize}
   \item contact mail
   \item to select a learning objective from a pre-formatted list
   \item to supply the question you thought of
   \item and (in case of a multiple choice question) the possible answers.
  \end{itemize}
  \pause
  \task[Evaluation Process]{Now, HPCCF-member evaluate the submitted question. If approved, it will be formatted and merged into the pool of questions for the choosen topic / skillset.}
\end{frame}



\end{document}

